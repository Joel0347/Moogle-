\section{Introducción}\label{intro}

\begin{frame}
    \begin{columns}[t]
        \begin{column}{.5\textwidth}
          \tableofcontents[sections={1-2},currentsection]
        \end{column}
        \begin{column}{.5\textwidth}
          \tableofcontents[sections={3-4},currentsection]
        \end{column}
    \end{columns}
\end{frame}

\subsection{¿Qué es Moogle!?}

\begin{frame}{¿Qué es Moogle!?}

\begin{center}
  Moogle! es una aplicación web cuyo propósito es buscar inteligentemente un texto en un conjunto de documentos. 
  Está desarrollada con tecnología .NET Core 7.0, usando Blazor como 
  *framework* web para la interfaz gráfica, y en el lenguaje C-Sharp.
\end{center}

\pause

  
\begin{center}
  La aplicación está dividida en dos componentes fundamentales:


  \ 


\textit{\textbf{MoogleServer}} es un servidor web que renderiza la interfaz gráfica y sirve los resultados.


\textit{\textbf{MoogleEngine}} es una biblioteca de clases donde está implementada la lógica 
del algoritmo de búsqueda.

\end{center}

\end{frame}


\subsection{Particularidades}
\begin{frame}{Particularidades}

\begin{itemize}

  \item El proyecto ha sido probado con un repositorio de alrededor de 15 mil documentos cuyo peso era de 170 MB.
  Con estos documentos el proyecto demora alrededor de 1 minuto inicialmente para cargar, y la búsqueda es 
  casi instantánea.
  \pause
  \item La búsqueda muestra los 10 primeros resultados como máximo. (Pudieran ser más si hay scores repetidos)
  \pause
  \item El buscador funciona tanto al seleccionar el botón “Buscar”, como al presionar la tecla “Enter”.
  \pause
  \item El snippet mostrado en pantalla de cada documento muestra una vecindad de la primera aparición de la
  palabra de más peso de la query con respecto a dicho documento.

\end{itemize}

\end{frame}

\begin{frame}{Particularidades}
  
\begin{itemize}
  
  \item Se ha implementado el operador de relevancia {*}, el operador de obligatoriedad {$\wedge$}, 
  el operador de inexistencia {!} y el operador de cercanía {$\thicksim$}. 
  \pause
  \item Los operadores {$\wedge$} y {!} funcionan solo para la primera palabra de la query que los 
  tenga. (Ej: “!hola !Claudia” solo funcionará para “hola”, mientras que a “Claudia” la tratará 
  como una palabra normal)
  \pause
  \item Pueden existir combinaciones de los operadores. (Uno en cada palabra distinta, no puede
  haber dos juntos en una misma palabra)
  \pause
  \item El operador de cercanía se debe usar de la forma: palabra1 {$\thicksim$} palabra2 (espacio entre 
  el operador y las palabras).

\end{itemize}

\end{frame}

